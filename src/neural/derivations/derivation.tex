\documentclass[12pt]{article}

\usepackage[margin=1in]{geometry}
\usepackage{amsmath,amsthm,amssymb}
\usepackage{graphicx}

\newcommand{\N}{\mathbb{N}}
\newcommand{\Z}{\mathbb{Z}}

\newenvironment{theorem}[2][Theorem]{\begin{trivlist}
\item[\hskip \labelsep {\bfseries #1}\hskip \labelsep {\bfseries #2.}]}{\end{trivlist}}
\newenvironment{lemma}[2][Lemma]{\begin{trivlist}
\item[\hskip \labelsep {\bfseries #1}\hskip \labelsep {\bfseries #2.}]}{\end{trivlist}}
\newenvironment{exercise}[2][Exercise]{\begin{trivlist}
\item[\hskip \labelsep {\bfseries #1}\hskip \labelsep {\bfseries #2.}]}{\end{trivlist}}
\newenvironment{reflection}[2][Reflection]{\begin{trivlist}
\item[\hskip \labelsep {\bfseries #1}\hskip \labelsep {\bfseries #2.}]}{\end{trivlist}}
\newenvironment{proposition}[2][Proposition]{\begin{trivlist}
\item[\hskip \labelsep {\bfseries #1}\hskip \labelsep {\bfseries #2.}]}{\end{trivlist}}
\newenvironment{corollary}[2][Corollary]{\begin{trivlist}
\item[\hskip \labelsep {\bfseries #1}\hskip \labelsep {\bfseries #2.}]}{\end{trivlist}}

\begin{document}

\title{Hypothesis for Neural Residual PDEs}
\maketitle

\noindent Define a dataset $\mathcal{D} = \{ \phi_i, \psi_i, \mathbf{x}_{i, 1:T} \}_{i=1}^N$ of boundary conditions $\phi$, initial conditions $\psi$, and a sequence of observations $\mathbf{x}_{1:T}$. We can define an observation in two ways: first, $\mathbf{x}_t = (\mathbf{u}_t, \mathbf{v}_t, \mathbf{p}_t)$ contains two dimensions of momentum and pressure; or second, $\mathbf{x}_t = (\lambda_t^{\mathbf{u}}, \lambda_t^{\mathbf{v}}, \lambda_t^{\mathbf{p}})$, a vector of spectral coefficients for momentum and pressure functions.\newline

\noindent Then, the function we want to learn consists of two components --- a base function $f \in \mathcal{F}$ where $f_\theta: \mathcal{X} \rightarrow \mathcal{X}$ over an observation. This base function is responsible for learning the dynamics over observations. Second, define a residual function $g_\theta: \{\phi, \psi\} \rightarrow \mathcal{F}$, tranforming boundary and initial conditions to a function. The base function is shared over all entries in the dataset --- we consider the following objective:
\begin{equation}
    \min \mathbb{E}_{\phi,\psi,\mathbf{x}_{1:T} \sim p_{\mathcal{D}}}\left[ \sum_{t=1}^T \mathbf{x}_t - (g_\theta(\phi,\psi) + f_\theta(\mathbf{x}_{t-1})) \right]
\end{equation}

\noindent Note that $g_\theta$ is kind of like a hypernetwork. Next question is what does $f_\theta$ look like (how does it relate to PDEs/ODEs)?

\end{document}